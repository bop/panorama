\documentclass[a4paper,12pt,openright]{memoir}
\usepackage[utf8]{inputenc}
\usepackage[T1]{fontenc}
\usepackage{url}
\usepackage{fourier}
\usepackage{fourier-orns}
\usepackage{verse}
\usepackage{lettrine}
\usepackage{graphics}
\usepackage[print]{pdfscreen}
\usepackage{psfont}
\pagestyle{plain}
\usepackage{aeguill}
\usepackage{times}
\usepackage{hyperref}
\hypersetup{
  bookmarksopen=true,		 % bookmarks are expanded 
  pdfpagelayout=SinglePage,
  pdfpagetransition=Dissolve,	% in full screen view only 
    unicode=true,          % non-Latin characters in Acrobat bookmarks
    pdftoolbar=true,        % show Acrobat toolbar?
    pdfmenubar=true,        % show Acrobat menu? 
    pdffitwindow=true,      % page fit to window when opened
    pdftitle={Le Point Commun},    % title
    pdfauthor={Baptiste Ossipow},     % author
     pdfcreator={LaTeX},
    pdfsubject={poetry},   % subject of the document
    pdfnewwindow=true,      % links in new window
    pdfkeywords={poetry}, % list of keywords
    colorlinks=true,       % false: boxed links; true: colored links
    linkcolor=black,          % color of internal links
    citecolor=green,        % color of links to bibliography
    filecolor=magenta,      % color of file links
    urlcolor=cyan           % color of external links
}

\textwidth=152mm
\textheight=240mm
\voffset -1in
%\hoffset 0in%-1in
\oddsidemargin=11mm
\evensidemargin=11mm
\topmargin=1.0cm
\footskip=1.3cm
\renewcommand{\ttdefault}{cmtt}
\usepackage[frenchle]{babel}
%\makeindex
\begin{document}

\definecolor{lightblue}{rgb}{.2,.5,1}
\newcommand{\lb}[1]{\color{lightblue} #1}

  \definecolor{section1}{rgb}{.2,.5,1}%{.650,.350, 0  }%{.937,.561,.123}
  \definecolor{bleu}{rgb}{.0,.0,.650}
  \newcommand{\bleu}[1]{\color{bleu} #1}

% Page de titre :
\pagestyle{empty}  
\title{\hbox{}
       \huge\bf
       \textit{Poésie}       
\\
  {\bleu Le Point Commun}
  \\ 
\author{\large\bf{ Baptiste Ossipow}}
\date{\today} }
\maketitle
\newpage
\epigraph{
  \hspace{5em}
  \vspace{7cm}
\\ \textit{`` Et la langue\\
  comme la terre\\
  se transmet ''}\\
\vspace{1em}
   {Mahmoud Darwich}}



\begin{verse}
\settowidth{\versewidth}{jamais vain creuset d'exil}
%\poemtitle{\bleu {Une étincelle}}
\vspace{1cm}
\chapter*{Le Point Commun}
\vspace{1cm}
{\lettrine[lines=1]{\textcolor[gray]{0.6}{P}}{ortée} par le courant marin}\\
les vagues leur écume atteignent\\
au pied des tours la terre ferme\\
l'océan et le ciel
 
\section*{}
les flots nous ont jetés\\
sur de puissants rivages\\
un continent se dresse\\
après l'insensé naufrage\\
où sombrèrent les autres

\section*{}
comment d'un chant les honorer\\
quel hommage rendre\\
l'aurore ébouriffée\\
sur la palette les lumières\\
des façades se recueillent\\
un instant pénètrent l'avenue

\section*{}
les naufragés le seuil franchissent\\
comme lance un nouveau jour\\
son filet dans la parole obscure\\
jamais vain creuset d'exil\\
la prière appelle puis scrute\\
parmi les pensées indistinctes\\
une présence à nos côtés

\section*{}
imperceptible ce qui descend\\
du dôme sous le vent s'élèvent\\
des ascenseurs d'or et de chrome\\
délaissent les rues inondées\\
des escalators nous déversent\\
que la rouille déjà dévore

\section*{}
les corsets de fers des parois\\
des mâchoires malaxent le sable\\
des éboueurs rincent le pavé\\
devant un bol de frijoles\\
j’attends jusqu'à qu’ils refroidissent

\section*{}
je regarde une jeune fille\\
comme une anguille se faufile\\
dans le smog entre les taxis\\
et le silence de son nom

\section*{}
nous épelons du raisin mûr\\
j'ai dévoré du fruit la pulpe\\
des perles en apesanteur

\section*{}
au paroxysme de l'échange\\
que ta langue a déployé\\
pour abolir l'esclavage\\
mélangeons-nous au soir sortons

\section*{}
l'acier dégouline poisseux\\
des chorégraphies enchaînées\\
au rythme lascif m'enlacent\\
serrant la braise entre ses grilles\\
le souffle nous soumet au feu

\section*{}
elle se maintient en offrande\\
dans la parole qui présente\\
tout peut être disposé\\
en nombres et en temps\\
afin que reste quelque chose\\
de ce soir le jour venu
\end{verse}
\vspace{2cm}
\begin{center}
\LARGE{\aldineleft}
\end{center}






%\appendix


\vfill


%\tableofcontents

%\\pagestyle{empty}

\begin{center}
\baselineskip=60pt 

\vspace*{12.5cm}
Réalisé avec \LaTeX
\\
\vspace*{1.5cm}
{\lb \LARGE \textcircled {bo}}

\end{center}
\end{document}
