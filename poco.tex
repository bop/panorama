\documentclass[a4paper,12pt,openright]{memoir}
\usepackage[utf8]{inputenc}
\usepackage[T1]{fontenc}
\usepackage{url}
\usepackage{fourier}
\usepackage{fourier-orns}
\usepackage{verse}
\usepackage{lettrine}
\usepackage{graphics}
\usepackage[print]{pdfscreen}
\usepackage{psfont}
\pagestyle{plain}
\usepackage{aeguill}
\usepackage{hyperref}
\usepackage{utopia}
\hypersetup{
  bookmarksopen=true,		 % bookmarks are expanded 
  pdfpagelayout=SinglePage,
  pdfpagetransition=Dissolve,	% in full screen view only 
    unicode=true,          % non-Latin characters in Acrobat bookmarks
    pdftoolbar=true,        % show Acrobat toolbar?
    pdfmenubar=true,        % show Acrobat menu? 
    pdffitwindow=true,      % page fit to window when opened
    pdftitle={Le Point Commun},    % title
    pdfauthor={Baptiste Ossipow},     % author
     pdfcreator={LaTeX},
    pdfsubject={poésie},   % subject of the document
    pdfnewwindow=true,      % links in new window
    pdfkeywords={poésie}, % list of keywords
    colorlinks=true,       % false: boxed links; true: colored links
    linkcolor=black,          % color of internal links
    citecolor=green,        % color of links to bibliography
    filecolor=magenta,      % color of file links
    urlcolor=cyan           % color of external links
}

\textwidth=152mm
\textheight=240mm
\voffset -1in
%\hoffset 0in%-1in
\oddsidemargin=11mm
\evensidemargin=11mm
\topmargin=1.0cm
\footskip=1.3cm
\renewcommand{\ttdefault}{cmtt}
\usepackage[frenchle]{babel}
%\makeindex
\begin{document}

\definecolor{lightblue}{rgb}{.2,.5,1}
\newcommand{\lb}[1]{\color{lightblue} #1}

  \definecolor{section1}{rgb}{.2,.5,1}%{.650,.350, 0  }%{.937,.561,.123}
  \definecolor{bleu}{rgb}{.0,.0,.650}
  \newcommand{\bleu}[1]{\color{bleu} #1}

% Page de titre :
\pagestyle{empty}  
\title{\hbox{}
       \huge\bf
       \textit{Poésie}       
\\
  {\bleu Le Point Commun}
  \\ 
\author{\large\bf{ Baptiste Ossipow}}
       }
\date{}
\maketitle
\newpage
\epigraph{
  \hspace{5em}
  \vspace{7cm}
\\ \textit{`` Et la langue\\
  comme la terre\\
  se transmet ''}\\
\vspace{1em}
   {Mahmoud Darwich}}



\begin{verse}
\settowidth{\versewidth}{jamais vain creuset d'exil}
%\poemtitle{\bleu {Une étincelle}}
\vspace{1cm}
\chapter*{Le Point Commun}
\vspace{1cm}
{\lettrine[lines=1]{\textcolor[gray]{0.6}{P}}{ortée} par les courants marins}\\
la vague le rivage atteind\\
au pied des tours la terre ferme\\
l'océan et le ciel
 
\section*{}
les flots nous ont jetés\\
sur de puissants rivages\\
le continent se dresse\\
après un insensé naufrage\\
où sombrèrent les autres

\section*{}
comment d'un chant les honorer\\
quel hommage rendre\\
l'aurore ébouriffée\\
sur la palette des lumières\\
des façades se recueillent\\
un instant pénètrent l'avenue

\section*{}
les naufragés le seuil franchissent\\
comme lance un nouveau jour\\
son filet dans la parole obscur\\
jamais vain creuset d'exil\\
la prière appelle puis scrute\\
parmi les pensées indistinctes\\
une présence à nos côtés

\section*{}
les escalators sous le dôme\\
les chromes rutilants s'élèvent\\
un ascenseur nous soulevant\\
imperceptible aussi descend\\
et tu vois les rues inondées\\
de la ville à travers les vitres
\end{verse}
\vspace{2cm}
\begin{center}
\LARGE{\aldineleft}
\end{center}






%\appendix


\vfill


%\tableofcontents

%\\pagestyle{empty}

\begin{center}
\baselineskip=60pt 

\vspace*{12.5cm}
Réalisé avec \LaTeX
\\
\vspace*{1.5cm}
{\lb \LARGE \textcircled {bo}}

\end{center}
\end{document}
