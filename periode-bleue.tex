\documentclass[a4paper, titlepage, twoside, 12pt]{book}

% Options possibles : 10pt, 11pt, 12pt (taille de la fonte)
%                     oneside, twoside (recto simple, recto-verso)
%                     draft, final (stade de développement)
%                     titlepage, notitlepage
%

\usepackage[babel,french=guillemets*]{csquotes}
%\MakeOuterQuote{``}
\frenchspacing
\usepackage[T1]{fontenc}      % Police contenant les caractères français

% bop => options perso
%
%%%%% épigraphe (poem's dedication)
%%%%%

\newcommand{\epigraph}[1]{\nopagebreak\afterpoemtitleskip\begin{epigraphquote}
\emph{#1}\end{epigraphquote}\afterpoemtitleskip\nopagebreak}
\newcommand{\headnote}[1]{\epigraph{#1}}
\newcommand{\attribution}[1]{\nopagebreak\afterpoemtitleskip\begin{epigraphquote}
{\small\emph{#1}}\end{epigraphquote}\afterpoemtitleskip\nopagebreak}
\newcommand{\poemdedication}[1]{\nopagebreak\afterpoemtitleskip\begin{epigraphquote}
\emph{#1}\end{epigraphquote}\afterpoemtitleskip\nopagebreak}
\providecommand{\dedication}[1]{\poemdedication{#1}}
\newcommand{\volumededication}[1]{\par\bigskip\begin{volumetitlepagequote}
\emph{#1}\end{volumetitlepagequote}}
\newcommand{\volumeepigraph}[1]{\par\bigskip\begin{volumetitlepagequote}
\emph{#1}\end{volumetitlepagequote}}

\usepackage[francais]{layout}

\usepackage{amsmath, amsfonts, amsthm}
\usepackage{fourier}
\usepackage{fourier-orns}
%\usepackage{kyril}
\usepackage{palatino, lettrine}
\usepackage{pdfpages}
\usepackage{verse}
\usepackage{url}
\usepackage{utopia}
\usepackage{aeguill}	% guillemets et ligatures francisés
%---PACKAGES-------------------------------------------------------------------

\usepackage{fancyhdr}

\cfoot{Page \thepage}
%\renewcommand{\headrulewidth}{1pt}
%\renewcommand{\footrulewidth}{0.4pt}

\usepackage{a4wide}
\usepackage[french]{minitoc}
\usepackage[hmargin=4cm,vmargin=3cm]{geometry}
\usepackage[frenchle]{babel}  % Placez ici une liste de langues, la
                              % dernière étant la langue principale

\pagestyle{plain}        % Pour mettre des entêtes avec les titres
                             % des sections en haut de page

%définition de la page de titre
 \makeatletter
\def\clap#1{\hbox to 0pt{\hss #1\hss}}%
\def\ligne#1{%
\hbox to \hsize{%
\vbox{\centering #1}}}%
\def\haut#1#2#3{%
\hbox to \hsize{%
\rlap{\vtop{\raggedright #1}}%
\hss
\clap{\vtop{\centering #2}}%
\hss
\llap{\vtop{\raggedleft #3}}}}%
\def\bas#1#2#3{%
\hbox to \hsize{%
\rlap{\vbox{\raggedright #1}}%
\hss
\clap{\vbox{\centering #2}}%
\hss
\llap{\vbox{\raggedleft #3}}}}%
\def\maketitle{%
\thispagestyle{empty}\vbox to \vsize{%
\haut{}{\@blurb}{}
\vfill
\vspace{1cm}
\includegraphics[width=12cm]{klee.jpg}
\begin{flushleft}
\usefont{OT1}{ptm}{m}{n}
\huge \@title
\end{flushleft}
\par
\hrule height 4pt
\par
\begin{flushright}
\usefont{OT1}{phv}{m}{n}
\Large \@author
\par
\end{flushright}
\vspace{1cm}
\vfill
\vfill
\bas{}{\@location}{}
}%              
\cleardoublepage
}

\def\author#1{\def\@author{#1}}
\def\title#1{\def\@title{#1}}
\def\location#1{\def\@location{#1}}
\def\blurb#1{\def\@blurb{#1}}
\date{\today}
\author{}
\title{}
\location{www.poetic.ch}\blurb{}
\makeatother
\title{ch\^ome }
\author{Baptiste Ossipow}
\location{www.poetic.ch}
\blurb{%
Po\&sie
}% 
                    % jour de compilation est utilisée en son
			      % absence
   %fin de la déf. de la page de titre
%\usepackage[frenchle]{babel}
\begin{document}
\maketitle                  % Faire un titre utilisant les données
                              % passées à \title, \author et \date

%\frontmatter                  % Prologue

\chapter*{chôme}
\newpage
%\mainmatter                   % On passe aux choses sérieuses
\pagestyle{plain}

\settowidth{\versewidth}{recouvert de poussière industrielle}
\poemtitle{La période bleue }
\begin{center}
\hspace*{5cm}
\textsl{pour Natasha}
\end{center}
%\poemdedication{à Nat}
\begin{verse}[\versewidth]
{\lettrine[lines=1]{\textcolor[gray]{0.6}{S}}{\space ur la ville} souffle coupé le vent}\\
de la poussière industrielle\\
un cheval apparaît la beauté sauvage\\
indomptée la crinière épiphanique\\
à l'affût de l'écoute\\
certains liens vont se nouer\\
des angoisses qui l'enserraient\\
à brasser l'air des milliers d'heures\\
l'âme est voyageuse pourtant\\
en marche d'ailleurs reconnue enfin\\
l'excursion sainte d'un atelier\\
apportant l'indispensable oxygène\\
puisque demeurer ici est impossible\\
entre l'appel et l'élection

\subsection*{}
mendiant leur argent pour manger\\
l'infirme devant le supermarché\\
en souffrir inouï sous l'offense\\
silhouettes à l'arrêt quand le soir\\
ils sont livrés à leur propre cupidité\\
sa gravité reprend une ombre\\
devenir furtive la énumération\\
peut-être les villes éloignent à jamais\\
de l'avoir vu là-bas\\
avoir lu dans le livre\\
que le montent à cru\\
ceux qui croient\\
ce serait une chevauchée\\
royale anachronique


\subsection*{}
les travailleurs se sont éteints\\
l'argent rugit\\
la rue est rouge\\
presque pourpre\\
l'alcool efface et coule\\
les moteurs retournent à l'inertie\\
dans le silence qui fascine\\
la rue est noire et l'ombre bleue\\
des enfants qui s'endorment\\
sur la mantille agenouillée\\
d'une vielle borgne\\
d'une provenance intouchable\\
une s\oe ur en guenilles\\
pour un peintre pose\\
son châle avec pudeur

\subsection*{}
l'inscription il est coupable\\
il doit l'être coupé à la racine\\
tranché de la cité\\
la ville où des noms à l'esprit s'ajoutent\\
l'opprobre ne subsiste aucune issue prescrite\\
dont on équerre les songes improductifs\\
mais partout éclaircir de certitudes froides\\
en bercer jusqu'à la fin tout se taise\\
derrière une lanière de peupliers\\
le non retour une barrière\\
un étalon se dresse\\
où les mots manquent disparaître

\subsection*{}
mais l'animal chétif veut vivre\\
inatteignable salve\\
frôler la peau dans son sommeil\\
touché la peur partir\\
au matin de l'exil\\
des chevaux paissent libres\\
sur le sol des sabots exaucent\\
au galop des rêves dansés\\ 
le souffle  des naseaux transporte\\
les muscles saillants des robes équestres\\
à son comble l'effroi\\
remis à l'éblouissement poétique\\ 
préservé dans l'attente\\
de la beauté du monde

\subsection*{}
sur la jetée de la nuit\\
le déferlement des villes entendent\\
transformé par l'écho\\
du mont descend le dict\\
humble est la puissance qui \oe uvre dans la lettre\\
la parabole créatrice\\
la parole donnée aux offensés\\
qui confiants remettent\\
à la grâce crucifiée du Christ\\
l'écoute lui prodigue\\
la langue retrouvée du temps\\
de la volupté où la horde chôme\\
si la splendeur se distancie\\
des mots d'amours qui l'ont reprise

\subsection*{}
de bas néons tamisent les ruelles\\
les flaques brillent le feu des huiles\\
les bras de verre se font enjôleurs\\
bouge l'immersion hallucinée sa fougue\\
des milliers d'un soir réconforte une amie\\
dans l'alcool oublie avoir combattu\\
un cheval dans l'azur où tout est destinal\\
de l'aube poignent les lueurs amniotiques\\
au lac des ablutions lavent le matin\\
à la terrasse d'un café encor ébahi\\
les passants par Pentecôte délivrés semblent\\
prolonger les discussions dans la cuisine\\
au visage de la faiblesse est apportée\\
la mémoire par cette eau vivifiée 

\subsection*{}
la puissance la volonté au jour données\\
l'énergie dans les entrailles préparatrices\\
cisèle attentive la dentelle sonore\\
mais la venue est certaine comme l'aurore\\
entretient les pimprenelles épineuses\\
devant la pensée l'animal avance\\
le col abaissé que tu puisses monter\\
sur la steppe acméiste enivrée d'espoir\\
sans bride en l'air portée\\
par la liberté d'aller où tu veux\\
laissant accomplir sa médiation\\
la langue dans son rapport conjugal\\
avec ce qui n'est pas


\end{verse}


\newpage


\settowidth{\versewidth}{dans le folklore un temps gardé en mémoire}
\poemtitle{Tomber}
\begin{verse}[\versewidth]
{\lettrine[lines=1]{\textcolor[gray]{0.6}{M}}{\space ouvement}} dans ta langue\\
rassemble à la mesure de la perte gravie\\
béance courageuse d'enjamber le nant\\
des orages reçoit l'Avançon ravissante\\
à l'âme humaine bien des choses le sont\\
déjà par leur précipitation disparaissent\\
dans le folklore un temps gardé en mémoire\\
des villages alentour à l'impact de foudre

\subsection*{}
les différenciations qui t'abaissèrent\\
voilà celles de hisser ton ascension\\
tu vois et l'étendue aussi qui nous attend\\
connaissance ces lois qui prennent soins des mots\\
inscrite la confiance en langue française\\
d'après l'émotion d'amour l'intellection\\
impérissable déjà l'entrevue éclore\\
mais dessous la croix pourtant les gammes d'étoiles\\
dans la démarche tonse sous les quolibets\\
porter la désolation libres rien être\\
pour magnifier la distance du retour\\
pour perdre le péril nous avons accompli\\
pour en prendre la mesure les conséquences\\
le désaveu dans la catastrophe nazie 

\end{verse}

\newpage

%\pagestyle{fancy}
\vspace*{2cm}
\settowidth{\versewidth}{que la terre accueille à ce qu'elle accorde}
\poemtitle{Hölderlin}
\begin{verse}[\versewidth]
{\lettrine[lines=1]{\textcolor[gray]{0.6}{\OE}}{\space uvre l'inconnue}} d'accueillir chez soi\\
vers l'indomptable langue matinale\\
la compréhension offerte porte du feu\\
où viennent les enfants habiter de leurs chants\\
dans son incalculable bonté la dire\\
ainsi c'est une telle célébration qui l'honore\\
les paroles du ch\oe ur que la mort\\
comme au bord du fleuve un promeneur\\
trouve du réconfort lorsque vient le soir\\
le vin d'or se laisse boire au flanc du jeune coteau\\
fruit du travail élevé à la célébration\\
la terre accorde à ce qu'elle accueille\\
la préséance saisonnière aux fruits de l'équité\\
où parviennent ça et là les roses\\
qui échappèrent à la logique sacrificielle\\
des éclairs grecs dans le minutieux dialecte\\
l'éclosion préservée pour la langue de l'adhérence\\
s'ils partaient dans les mythes traduisaient en science\\
l'art voyage respectueux la beauté à distance\\
assembler dans les rues encor joyeuses de Tübingen\\
qu'au retour de Gironde il soit gardé la mémoire\\
à Francfort Suzette qu'il aimait s'est éteinte\\
dans les chants en commun de l'hospitalité pensive\\
\end{verse}
\subsection*{}
\settowidth{\versewidth}{que tout soit comme les foins}
%\poemtitle{ }
\begin{verse}[\versewidth]
une danse russe importe\\
à la fenaison venue\\
l'ouvrage est fait d'attention\\
mise au labeur de l'exil\\
que tout soit comme les foins\\
aux émondements vouloir\\
les folioles du trèfle\\
des constellations qu'embrasse\\
l'alpe du pays d'en-haut\\
l'aliénation muselée\\
sitôt l'oublieuse foudre\\
déjà en Grèce l'esprit\\
est l'art que sa trace inscrit\\
en traversant ses ténèbres\\
le noir Christ des lumières\\
au-dessus de la charpente\\
le panorama autour\\
de la nature amoureuse\\
l'écoulement de l'eau du fleuve\\
l'effluve printanière\\
dialogue dans la chambre\\
l'image des ciels futurs\\
des escaliers le Neckar\\
le destin des arabesques\\
de fumée est disparaître\\
sans considération\\
au matin de chaque jour\\
l'indécision séminale\\
des antiques retrouvailles\\
nous réserve des soleils\\
aux ravages infinis\\
dans une Allemagne cubiste\\
la culmination globale\\
un psaume pour le réveil\\
le corps déclame un poème\\
au matin de chaque jour
\end{verse}


\newpage

\vspace*{2cm}
\settowidth{\versewidth}{la toile sur la résurrection lever}
\poemtitle{Roublev}
\begin{verse}[\versewidth]
{\lettrine[lines=1]{\textcolor[gray]{0.6}{D}}{\space es plaines sémitiques}} au nord de l'Afrique

la toile sur la résurrection lever\\
l'enfant diaphane dans les bras magnifiques\\
le sang coule la neige enrobant l'alentour\\
pour que j'ose te déranger par des prières\\
sous l'indistincte violence de la guerre\\
menée contre la liberté de notre foi\\
qu'avec nous le dieu des origines dialogue\\
il est écrit dans les livres de la Torah\\
le cancer de la peur qui dévore nos villes\\
et les folles dîme qui les rendent exsangues\\
ils firent à Ghomore ils font aux étrangers\\
puissiez-vous contempler l'icône de Roublev\\
sous le chêne lointain de la visitation\\
où l'ange du retour rédempte le fauteur 
\end{verse}
\newpage


\vspace*{2cm}
\settowidth{\versewidth}{les épaules lourdes corrections de la nuque}
\poemtitle{Djihad}
\begin{verse}[\versewidth]
{\lettrine[lines=1]{\textcolor[gray]{0.6}{O}}{\space raisons d'architecture}}  la passion\\
les épaules lourdes corrections de la nuque\\
aux piloris portugais des amitiés\\
sans pitié septembre déjà pointe son nez\\
quoi qu'en disent Genevois c'est un bel été\\
nous asservir sous la propagande assénée\\
par les vertus ondoyantes de l'air vibré\\
interrompt ajustés en silence exorbitant\\
à Salamanque l'obscurité de la nuit\\
les exilés juifs de ton vocabulaire\\
pour le chant de grâce monte parfois le soir\\
de l'affliction côtoyée la permanence 
\end{verse}

\newpage


\vspace*{2cm}
\settowidth{\versewidth}{que ces chemins menent quelque part}
\poemtitle{Chemins qui mènent quelque part}
\begin{verse}[\versewidth]
{\lettrine[lines=1]{\textcolor[gray]{0.6}{I}}{\space mpasse}} des lieux retrouvés\\
le penseur étoilé quitter la Lituanie\\
dérive prise dans la litanie de glace\\
impasse aussi des lieux fuis
\subsection*{}
si l'information s'automatise\\
les boulons des machines on doit resserrer\\
à vélo le matin je fends l'air vers l'usine\\
l'uniforme imposé sur la peau\\
\subsection*{}
si tôt le matin nous empruntons ces chemins\\
ce serait comme la permanence affectueuse\\
aux ailes d'une impression endurée\\
que ces chemins mènent quelque part
\end{verse}


\newpage
\vspace*{2cm}
\settowidth{\versewidth}{éclore en nous l'effroi de ne pas encore être}
\poemtitle{Gen 2.15}
\begin{verse}[\versewidth]
{\lettrine[lines=1]{\textcolor[gray]{0.6}{L}}{\space a parole donne}}  irruption confiante\\
notre immigration dans le jeu du langage\\
lorsque l'année s'achève autour du corps souffrant\\
que vient son étonnante résurrection\\
éclore en nous l'effroi de ne pas encore être\\
accomplissant sans adhérence dévaster\\
tandis que tout autre devoir en résistance\\
partage l'étendue insurrectionnelle\\
à l'avant du chant dans les flots messianiques\\
l'heure où nous boirons à la source poétique\\
prodiguant l'énergie atomiques fissions\\
à l'effacement seul la trace que subsiste\\
établir le droit qui assure la justice\\
leur affranchissement aux offensés remettent\\
le vent des hauteurs infirmées de la douleur\\
est tombé j'en témoigne sur l'adoration\\
l'acquiescement au dessein de leur dieu\\
les ravages subir tout le jour en son nom\\
sans imposition ils ont choisi le souffle\\
dans le soigneux argile du corps son haleine\\
"\emph{pour le servir et le garder}" dit le Livre 
\end{verse}

\newpage
\vspace*{2cm}
\settowidth{\versewidth}{la dissipation les brumes aveugles de quelle nuit}
\poemtitle{À la Nuit}
\begin{verse}[\versewidth]
{\lettrine[lines=1]{\textcolor[gray]{0.6}{A}}{\space près}} l'effroi\\
survenue dans l'abaissement\\
la dissipation les brumes aveugles de quelle nuit\\
où elle nous a conduits\\
des égarements philosophiques complices\\
après ça\\
la lente dissipation partage\\
retenue à la blême chevelure du temps\\
à la noire rémanence des illuminations\\
mutique Nuit\\
couverte de crimes sans châtiment\\
recouvrant l'agonie de consciencieux silences\\
le fardeau déporté des crimes d'état\\
la science mise au service des meurtriers\\
où s'élèvent les fumées sacrificielles\\
dans le ciel blanc d'un été sans soleil\\
au nom des lois raciales qu'elle invente\\
ceux qui échappent sont conduits aux frontières\\
des enfants leur père les suggère au suicide\\
sont remis en fait à la liberté de leur dieu\\
l'insurrection cubiste radicalité\\
sous les coupes sociales des mantes officient\\
prennent ensuite grand soin de leur victime\\
quotidiennement foulée se côtoient\\
marcher sur la tête de l'\'Etranger\\
ai-je souhaité que en corps tu fusses là
\section*{}
ne croyant désormais plus en rien\\
l'inquiétude devant tout ce qui vient\\
conjurée à perte de vue\\
transhumance glaciale\\
que prennent des formes insensées de nous\\
nous abaissons aux accoutumances des drogues\\
des lotus l'âme et le corps scinde léthargie\\
passer le cap l'alcool et les calmants\\
le bourdonnement du silence oppression dessaisir\\
nous maintenant\\
notre main tenant\\
de la retenue l'appel\\
recevoir l'expiation pour le détenu\\
qui m'avait trouvé pourpre de honte esquisser\\
maladresse incommensurable l'historique fresque\\
flèche inversée sur la croix du clocher descendent\\
avec le seigneur des agapes bénissantes\\
des mouvements délivrés prennent corps du matin\\
requérant avant la cadence laborieuse\\
dans l'allée les  néons le samedi chez Aldi\\
l'étal déroule la sommation d'acheter\\
les enseignes du remboursement prédominent\\
la sélection productive impitoyable\\
ses critères sont mêmes qui menaient au pire\\
--- mais bien sûr ceci rentre en ligne de compte\\
le diable aimerait que nous aussi --- enfin bref\\
table conviviale faire\\
fendre la foule hilare de la rue du calvaire\\
quête éthique pour après le capitalisme\\  
quand du doute sonne l'heure\\
il reste la langue dans laquelle je pense\\
qui est avec nous\\
dans laquelle joue\\
une autre logique\\
l'épître de Jean\\
tire les leçons d'amour\\
Augustin dit qu'il autorise tout\\
en restituer la terre\\
où tout n'est pas cupide\\
parfois nous allions\\
en musique à l'usine\\
des vieilleries aux dernières mélodies\\
et nous allons\\
contre l'autre nos corps\\
à ton inattendue saveur\\
sous les dépaysements te transporte\\
un silence dans le vide\\
troué de nous\\
selon la grâce incarnée\\
sur tes lèvres un soir de juin ne se dérobent\\
c'est le bonheur à deux un moment de folie\\
dès lors l'un pour l'autre de loin l'évoquer\\
les coups bas du sort ni les ans n'effacent\\
icône ouverte vue d'autres analogies\\
la souterraine porosité un film dépose\\
tes traces ne se décelaient pas d'alentours\\
la plaine au crépuscule baigne sa lueur\\
la clameur continue prodigieuse dépense\\
l'énergie à peine extraite son déferlement\\
des certitudes s'évidant en éclats volent\\
si le mutisme est lié au déracinement\\
le fleuve découle alors d'une analogie \\
avec la conscience Danaé chevauchée \\
des stalagmites croissance recueillie\\
la foi transfigurant le dragon de la crainte\\
un art des falaises sous une bonne étoile\\
telle confiance est accordée rémission\\
la porte les mots le sanctuaire du corps\\
peinant à l'année la prononciation éclore\\
des bribes arrachées aux bêtes imposées\\
qu'à l'écoute de l'inconnu on ne pense\\
amenée par les propositions de la langue\\
vivant maintenant à distance migratoire
\section*{}
plus clairs tourments du ciel flamand\\
quand flottent les nappes au vent\\
la dentelle s'agite des fenêtres\\
sous une aubade au ras de la chaussée\\
effacé dans le décors étreinte douleur\\
dehors le dernier établissement ouvert\\
l'infirme est chassé sans le gîte ni couverts\\
survolent les goélands de la plage frises\\
d'écumes courbe une silhouette pourtant\\
approche avec faste le verbe du repos\\
sur les calabères à figues son regard\\
mais le négatif aussi à l'image de\\
saveurs du livre autour des mots volés par c\oe ur\\
ils sont si légers qu'ils naissent l'éternité\\
pour poser sur des lèvres paroles passagères\\
unique traversée de permanence inscrite\\
si libres des vers luisent de l'aube\\
issus des bonds entre les grappes le soleil\\
venu sur les terrasses pavées de septembre\\
l'ombre des toits d'Amsterdam étendue apporte\\
de la main dépose avec la grâce des fleurs\\
le refuge longeant les canaux de l'insomnie\\
la fraîcheur matinale vienne saluer\\
à moins que ne reçoive l'infirme secours\\ 
qu'est la miséricorde pour le misérable \\
l'expiation supportée de nos péchés\\
l'ouvrage accompli des jours dit dans le secret\\
sa quotidienne déconsidération\\
mais l'infirme avancé dans la nuit sans peur\\
touche le vent désaffecté avec la paume\\
le nom de chaque étoile connue est Seigneur\\
bâtir étonnant en Jérusalem un psaume\\
pour le repos du verbe les bannis rassemble\\
et les enfants dans des églises qu'on étrangle\\
dehors les anges aux néons brûlent leurs ailes\\
couronnement de mort leur missive perdue\\
redressement silencieux de ce qui est tordu\\
sans consumer les dévore un étrange zèle\\
préparer les rubans pour l'arrivée du roi\\
résonner sa parole inviolée en moi\\
du refus opposer mendiant une obole\\
du rêve quadrillé que font les métropoles\\
les mathématiques prennent vie sur écran\\
les représentations d'arborescents calculs\\
en procession fragmentés jusque n'être que\\
nous les maîtrisions aux ténèbres de l'oubli\\
mais la toile des femmes chevalet cubisme\\
l'harmonie la méprise révolte sociale\\
la cité appentie en face du Christ vit\\
le dénuement poétique à la radio\\
plus en récolte là sa générosité\\
ainsi les mots donnés un sens entre deux seuils\\
à la pureté de la forme rupture portée\\
derrière l'angoisse au milieu de la pensée\\
l'océan librement infini de la langue\\
nous recueillons les propositions amoureuses\\
que les étoiles adressent à la Nuit\\
nous répandons dans le silence une semence\\
et le chant que notre vie féconde un instant\\
la sève en esprit adorable au firmament

\end{verse}
\newpage

\vspace*{2cm}
\settowidth{\versewidth}{aux pieds des versants rejoints dans le val}
\poemtitle{Chemins de fer}
\begin{verse}[\versewidth]
{\lettrine[lines=1]{\textcolor[gray]{0.6}{Q}}{\space ue le hall}} de gare est bondé\\
la file d'attente aux guichets\\
le train fait son entrée à quai\\
la masse métallique glisse\\
les essieux crissent en freinant\\
les passagers laissent leurs songes\\
et des journaux sur les banquettes\\
\subsection*{}
un flux d'air climatisé me frigorifie\\
indigne battue la sélection naturelle\\
livre l'infirme projette sa fin\\
dans le couloir où nous cherchons salaire\\
le rail sort de la ville\\
aux pieds des versants rejoints dans le val\\
les Voyelles de Rimbaud\\
\subsection*{}
le vertige des lettres qu'inonde l'empreinte\\
sur des pages plus libre ascension du sommet\\
blancheur surmontée l'attraction de la pente\\
de les dire est assez aux mots donner la vie\\
prier pour sous la tente nous mettre à l'abri\\
autour des choses auxquelles nous pensons\\
\end{verse}

\newpage

\vspace*{2cm}
\settowidth{\versewidth}{ou la langue est en dieu}
\poemtitle*{}
\addcontentsline{toc}{poemtitle}{ { } La Cité des Poètes
. . . . . . . . . . . . . . . . . . . . . . . . . . . . . . . . . .  }
\begin{verse}[\versewidth]
{\lettrine[lines=1]{\textcolor[gray]{0.6}{L}}{\space a Cité des poètes}}\\
où la langue est en dieu\\
dieu qui nous parlerait\\
et nous entendrions
\subsection*{}
que la mort est vaincue\\
dans son abaissement\\
nous combleraient d'amour
\subsection*{}
ce n'est pas la coutume\\
les propositions
\subsection*{}
d'une adhérence libre

\end{verse}

\newpage

\vspace*{2cm}
\settowidth{\versewidth}{comme le jour est long quarante ans dans les déserts}
\poemtitle{L'Exode}
\begin{verse}[\versewidth]
{\lettrine[lines=1]{\textcolor[gray]{0.6}{J}}{\space uché sur les hauteurs}} arrogantes qui dominent\\
le présent est une sensation fugace\\
dans laquelle tout ne vient pas se traduire\\
ce qui est revêt une telle profondeur\\
resterait-il une place pour rien d'autre
\subsection*{}
comme le jour est long quarante ans dans les déserts\\
livrés à la quête d'absolu absolue\\
Moïse choisi de l'horrible angoisse hors\\
le peuple hébreux mis à l'avant de la mer Rouge\\
des peuples sortirent pour aller adorer
\end{verse}


\newpage

\vspace*{2cm}
\settowidth{\versewidth}{cristaux infimes variations de temperature}
\poemtitle{Une humble présence}
\begin{verse}[\versewidth]
{\lettrine[lines=1]{\textcolor[gray]{0.6}{Q}}{\space uel hiver}} survivre l'état de frénésie\\
laissant ce qui aurait pu naître\\
emporté par les filandreuses méduses des rêves\\
devant l'histoire vaincue la force événementielle\\
semble ne laisser rien d'autre que des peut-être\\
mais ne les préférons-nous pas encor\\
l'amour n'est pas sous la contrainte il est l'attente\\
de son humble puissance mise en nous\\
celui à terre pourtant se relève\\
si les têtes dépassent les coupent\\
on fait ce qu'on ne veut subir aux enfants\\
la chasse aux déviants dans la cité idéale\\
ils se répartissent en petits groupes par quartiers\\
ce qui prouve bien leur malhonnêteté\\
légalement la meute se lance à leur trousse\\
nous consentions à l'horreur perpétuée\\
on s'acharnera sur ceux qui tombent\\
par dressage systématique on nous assigne\\
les plus forts devant dirigent l'espèce\\
à qui on doit la survie dans l'hostilité\\
on se prosterne donc également à leurs pieds\\
quand la servilité s'ajuste au nécessaire\\
on en ressort abasourdi de violence
\subsection*{}
à l'écart les ténèbres ont emparés la nuit\\
entre le fait et ses possibles résonances\\
nous errons libres où nous leur devons rien\\
et libre est l'air autour de ton visage aussi\\
et avec la parole une forme je lui donne\\
calmement à l'esprit qui vient dans la langue\\
mystérieusement approuver la beauté de l'\oe uvre\\
la justice rendue à l'enfant à l'infirme\\
comme auprès ceux qui souffrent son humble présence\\
le seigneur du ciel et de la terre avec nous\\
si nos mérites certes sont étourdissants\\
d'ailleurs révèle l'harmonie qui nous dépasse
\subsection*{}
il neige\\
sur la ville\\
mon poème\\
blanc linceul sabbatique\\
affranchissement l'année\\
cristaux infimes variations de température\\
l'Europe couverte les montagnes immobiles\\
descendent des navires sur la mer\\
sur la matière s'exerce notre pouvoir\\
mais l'esprit nous pouvons le renier\\
délier l'inscription demeure dans la langue\\
confiante colonne de verbe
\subsection*{}
don de la foi en un autre\\
que l'âme a reçu des capacités langagières\\
acquiescer à la venue initiale\\
que creuse la présence sans certitude\\
le refus proposer au néant\\
exigence joie d'exister\\
portant les consolations éternelles\\
au calice de mes lèvres en sang\\
le soin pris qu'il ne se renverse\\
les lettres de la tora insuffle l'ange du livre\\
de tous les livres que tu lis\\
et que tu reposes comme un trésor en toi
\subsection*{}
voici le ravissement dans le psaume des degrés\\
porté par les vagues de plaisir\\
qui découvre aux regards des amoureux\\
entre les branches juvéniles d'un vieux chêne\\
une blanche colombe saigne abondamment\\
l'alphabet primordial aussi a une tache rouge\\
le point d'équilibre des forces fondatrices\\
que les héritiers de la promesse déplace\\
pour elle inscrire ainsi sa promesse\\
le trait des glyphes sur le chemin historique\\
qu'intégralement soit attesté le règne\\
du descendant de David en pérennité
\end{verse}


\newpage

\vspace*{2cm}
\settowidth{\versewidth}{advienne quelque chose d'autre}
\poemtitle{Chorégraphie}
\begin{verse}[\versewidth]
{\lettrine[lines=1]{\textcolor[gray]{0.6}{C}}{\space ubisme des mots} le temps d'écrire}\\
comme ils fusent dans le ciel de la pensée\\
un poème prendre mon corps\\
en transition incandescente\\
entre l'énergie fusionnelle\\
le magma contraint du corps de la terre\\
et dans l'air l'haleine des voix\\
éclosent libres et chantent
\subsubsection*{étude de mouvement}
prenant vampire non abysse\\
que tous les possibles épuisent\\
dans le mot un vers expérimente\\
au solstice définitif à travers\\
une rime une association nouvelle\\
que nous partageons entre les langues\\
chantées sur le seuil nuptial\\
la communion des noms\\
j'ose laisser hisser où\\
l'écoute silencieuse de la pensée amène\\
dans les effluves d'une goutte de l'esprit\\
le calme produit de la permanence éveille\\
dans la chair tremblante le rire\\
de jeunes danses s'ensuivirent
\subsubsection*{fantaisies}
les premiers trouvèrent\\
la terre ferme n'en revinrent\\
les colombes font retour seules
\subsubsection*{}
être être présent là\\
c'est l'évidence même\\
et la grandeur froide sans doute\\
des syllogismes de l'antiquité\\
cependant les choses parfois\\
ne se passent pas comme prévu\\
ce qui peut aussi faire rire\\
advienne quelque chose d'autre
\subsubsection*{}
des étoiles la touchent\\
la peau nue sous la douche\\
le temps est à l'écoute\\
et les gouttes de houle\\
suspendues un instant\\
un étincellement
\subsubsection*{}
le temps recueillement du cycle saisonnier\\
purifier la langue de la foi\\
l'émotive diction\\
et le vent éloquent souffle d'été\\
l'infime brise du poème

\end{verse}


\newpage


\vspace*{2cm}
\settowidth{\versewidth}{qui dessine noir au-dessus les Alpes}
\poemtitle{Flore}
\begin{verse}[\versewidth]
{\lettrine[lines=1]{\textcolor[gray]{0.6}{T}}{\space ension} végétale}\\
irrépressible poussée\\
sous la terre mutique du retour\\
malade dans l'attente Europe\\
soleil en colonie des langues cultivées\\
liberté du partage la parole\\
qui dessine au-dessus les Alpes\\
des rivages lumineux les mers\\
les bouquets d'analogies\\
la participation aux offrandes comprises\\
partout fleurit de la poésie\\
l'émotion commune à la racine du nom 

\end{verse}



\vspace*{1.5cm}
\begin{center}
\LARGE{\aldineleft}
\end{center}  




\newpage
\tableofcontents            % Table des matières
\newpage
\pagestyle{empty}

\begin{center}
\baselineskip=20pt 

\vspace*{17.5cm}
Baptiste Ossipow \'Editeur 2012
\\
\vspace*{1.5cm}
\url{www.poetic.ch}
\\
\textcircled{bo}
\end{center}
% \listoffigures              % Table des figures

% \listoftables               % Liste des tableaux

\newpage

\layout


\end{document}
